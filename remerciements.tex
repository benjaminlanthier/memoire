\begin{comment}
\end{comment}

% Merci!

Je voudrais commencer par remercier Stefanos pour m'avoir accepté dans son groupe de recherche qui, à prime abord, ne semble pas trop connecté avec mes études en ingénierie physique.
Tu m'as donné un projet super intéressant, mais tu as aussi su être un excellent guide tout au long de mon parcours.
Merci pour ta patience et de m'avoir mis en confiance, tant avec les connaissances théoriques requises pour ce projet qu'avec tes conseils au niveau du code.

Merci à Jeremy pour ton aide précieuse tout au long de ce projet, tu as su être un excellent mentor et ton implication a été grandement appréciée.
Tu as toujours été disponible pour répondre à mes questions, peu importe leur niveau de difficulté, mais aussi pour nos discussions sur nos objectifs et motivations.
Ça m'a fait grand plaisir!
Merci aussi à tous les membres du groupe pour votre aide, votre écoute, mais aussi pour les moments que nous avons partagés et les liens que nous avons formés.
Que ce soit les parties de rugby dehors durant l'été, les séances de frisbee, ou les conférences, nous avons su créer une belle ambiance qui a aidé à alléger la pression sur nos épaules, et j'en suis très reconnaissant.
Je suis chanceux d'être tombé dans un groupe avec une ambiance comme la nôtre.

Je tiens maintenant à remercier toute ma famille pour vos encouragements ainsi que votre soutien constant.
Vous apportez toujours de la joie dans mes journées lorsque nous sommes ensemble, et je vous en serai éternellement reconnaissant!

Un grand merci à tous mes amis qui, par un heureux hasard à la Polytechnique, se sont retrouvés dans le même bateau que moi pendant ces deux années.
Béatrice, Dominic, Jonathan, Marie-Frédérique, Thomas, Thomas et Simon; vous avez été une source de connaissances, de motivation, de niaiseries et de sorties — bref, vous avez rendu ces années mémorables!
Je suis particulièrement reconnaissant que \textit{niaiseries} et \textit{sorties} soient au pluriel dans cette phrase, car tous ces moments ont égayé des journées qui, par moments, pouvaient sembler plus grises pendant cette période de recherche.
Un merci tout spécial à Laurence (et son Léo) pour ton support incroyable, ta patience et ta bonne humeur contagieuse!
Tu m'as motivé et encouragé dans des moments plus difficiles et ça, je t'en suis redevable!

Finalement, je voudrais remercier, de manière générale, tous ceux que j'aime!
Sans vous, je ne pense pas que je serais rendu où je suis en ce moment.
Vous avez été les phares qui m'ont guidé à travers le brouillard de ces deux années de maîtrise, mais aussi le Soleil qui me permettait d'entrevoir la fin de ce projet.

% Julien, Martin, Antoine, Pierre-Alexandre, Jérémie et j'en passe pour m'avoir écouté et aidé lors des passages plus difficiles, mais aussi pour tous les moments qu'on a passés ensemble.
% Que ce soit les parties de rugby dehors durant l'été, les séances de frisbee ou les soirées un peu plus arosées, on a su créer une belle ambiance entre nous qui aide à baisser la pression qui nous est mise sur les épaules et j'en suis tr;s reconnaissant.

% Évidemment, un grand merci à Stefanos pour m'avoir parlé de ce projet qui a grandement piqué ma curiosité au début de la maîtrise et de m'avoir partagé ses connaissances sur les sujets qui se retrouvent dans ce projet.
% La première fois que je suis allé à l'UdS, c'était avec des amis pour voir les sujets de recherche des professeurs en physique.
% Je ne me doutais pas du tout que j'allais trouvé un groupe de recherche qui m'intéresse, encore moins qu'il m'intéresse assez pour m'inscrire à une maîtrise, mais me voici!
% Sa présentation sur l'utilisation de méthodes développées pour la physique qui peuvent être utilisées dans d'autres domaines scientifiques a piqué ma curiosité.

% Maintenant, je tiens à remercier tous mes amis de Poly --- MF, Doom, Tom, Jo et Simon ---, des personnes incroyables avec qui tout peut être un sujet de dicsussion, tant la physique que les sujets les plus randoms comme les pénis quantiques.
% Sans vous, ce deux ans auraient paru beaucoup plus long qu'il ne l'était vraiment!

% J'ai aussi un merci spécial pour toutes les merveilleuses personnes que j'ai rencontrées à l'université; Ben, Étienne, Luca, Julien, Émile.

% Pour finir, je tiens à remercier toute ma famille pour m'avoir soutenu et encouragé durant ce parcours.
% Votre constant soutient ainsi que vos encouragements sont ce qui m'a permis de pousser plus loin même lorsque la montagne semblait trop haute!
% Aussi, discuter des sujets qui se trouvent dans mon projet avec vous et vous voir vous informer pour essayer de me suivre lorsque vous me demandiez ce sur quoi je travaille, ça a été un vrai plaisir!% Vos tentatives d'écoutes de mon projet me touchent et, sincèrement, vous m'avez aidé à régler certains de mes problèmes alors vous m'avez permis d'avancer!


