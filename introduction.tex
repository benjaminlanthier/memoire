\begin{comment}
\end{comment}

\Introduction   % Chapitre qui ne sera pas numéroté si IntroConcluSansNombre est Vrai

%-----------------------------------------------------------------------------%
Les verres de spin sont une classe de matériaux magnétiques caractérisés par leurs interactions désordonnées et frustrées entre les moments magnétiques individuels (spins)~\cite{stein2013spin}.
Contrairement aux ferromagnétiques ou aux antiferromagnétiques traditionnels où les spins s'alignent de manière uniforme, les verres de spin en présentent un arrangement complexe et aléatoire.
De nos jours, la physique des verres de spins se retrouve dans des domaines qui sont assez éloignés de son domaine d'origine, la matière condensée, comme dans l'informatique théorique~\cite{kirkpatrick1985configuration}, la biologie~\cite{bryngelson1987spin}, l'apprentissage machine~\cite{venkataraman1991spin} et même la planification des horaires pour les compagnies en aviation~\cite{stein2013spin}.
Les modèles de ces systèmes physiques sont, de manière générale, faciles à décrire, mais difficile à résoudre, donc difficile de trouver un état fondamental.
La principale explication de ce défi est que ces modèles présentent des paysages énergétiques qui ont une allure accidentée~\cite{stein2013spin, qing2009energylandscape}, ce qui coince les algorithmes d'optimisation dans des minimums locaux et cause un temps de résolution exponentiellement long.

Un contre-exemple de ces modèles est le $p$-spin~\cite{mezard_alternative_2002}, qui est en fait facilement résolvable même s'il possède le même genre de paysage énergétique que les autres modèles dans certains régimes de paramètres.
Effectivement, les systèmes modélisés par ce modèle peuvent se traduire en un système d'équations linéaires modulo $2$ et l'élimination gaussienne classique nous permet d'obtenir la fonction de partition du système à température nulle dans un temps polynomial.
Même si ce modèle est une version restreinte d'un modèle général de verre de spin, son analyse fournit des informations utiles sur la physique des verres de spin.
Comme il montre le même genre d'évolution énergétique en fonction des combinaisons de spins possibles et qu'il est simple à résoudre, il est aussi une référence standard pour les algorithmes classiques~\cite{bernaschi2021we, kanao2022simulated, aadit2023all} et quantiques~\cite{jorg2010first, farhi2012performance, hen2019equation, bellitti2021entropic, kowalsky20223, patil_obstacles_2019}.
Dans les régimes dans lesquels ce modèle possède le même genre de paysage énergétique, le recuit simulé n'est pas efficace, ou échoue, pour les systèmes pour lequel $p > 2$~\cite{patil_obstacles_2019, bellitti2021entropic} et c'est aussi vrai pour la méthode de recuit simulé quantique~\cite{farhi2012performance} même si aucune transition de phase n'est rencontrée~\cite{patil_obstacles_2019}.
La satisfiabilité booléenne et même les serveurs locaux ont de la difficulté avec ces modèles~\cite{haanpaa2006hard, jarvisalo2006further,jia2004spin, barthel2002hiding, ricci2001simplest, guidetti2011complexity}.

Ce mémoire introduit une méthode qui permet de résoudre les modèles $p$-spin en utilisant la satisfiabilité booléenne ainsi que les réseaux de tenseurs.
Les réseaux de tenseurs sont une structure de donnée qui permet une représentation compacte d'un modèle graphique (pondéré ou non), comme les Hamiltoniens de spin et les problèmes de satisfaction de contraintes, qui reviennent à compter le nombre de solutions au problème donné~\cite{garcia-saez_exact_2011, biamonte2015tensor, kourtis_fast_2019, Meichanetzidis_2019, de_beaudrap_tensor_2021}.
Bien que la contraction de réseaux de tenseurs soit, de manière générale, numériquement difficile même pour des classes de graphes restreintes comme les grilles planaires~\cite{schuch2007computational}, les techniques impliquant la compression de tenseur peuvent conduire à un résultat approximatif précis des fonctions de partition classiques ou des valeurs moyennes en mécanique quantiques (dans des cas spécifiques) de manière efficace.
Les réseaux de tenseurs ont aussi été utilisés dans des études de champs moyens de modèles graphiques et d'Hamiltonien de spin~\cite{alkabetz2021tensor, pancotti2023one}.

Dans la partie~\ref{part:theory} sur la théorie, les verres de spin ainsi que le modèle $p$-spin sont présentés dans les sections~\ref{sec:spin-glass} et~\ref{sec:p-spin} respectivement.
Elles sont ensuite suivies par une introduction sur la satisfiabilité booléenne à la section~\ref{sec:def-SAT} et par la définition du problème général \#$k$-XORSAT, qui permet la traduction d'un système $p$-spin à ce problème fondamental en informatique théorique, à la section~\ref{sec:k-xorsat}.
Les réseaux de tenseurs finissent ensuite le bal au chapitre~\ref{ch:TN}, où cet outil numérique ainsi que certaines de ses méthodes sont présentées.
Ensuite, la partie~\ref{part:methodology} sur la méthodologie explique comment relier le problème de satisfiabilité aux réseaux de tenseurs à la section~\ref{sec:tn-for-p-xorsat} et comment optimiser la contraction de ces réseaux à la section~\ref{sec:eliminate-redundancy-with-bond-compression}.
Elle contient aussi le chapitre~\ref{ch:graphical-method}, qui explique une méthode graphique développée entièrement afin de pouvoir analyser un ensemble de connexion spécifique des systèmes $p$-spins, ainsi que la méthode d'implémentation numérique de l'algorithme développé ce projet au chapitre~\ref{ch:numerical-implementation}.
Finalement, les résultats des simulations sont présentés dans la partie~\ref{part:results} pour les ensembles analysés numériquement et ceux analysés de manière graphique, aux sections~\ref{sec:results-random_instances} et~\ref{sec:results-only_cores} respectivement et des améliorations possibles de l'algorithme développé sont discutées au chapitre~\ref{ch:possible-ameliorations}.

% Here, we show that compressed TN contraction applied to the $p$-spin model automatically emulates previously discovered algorithms for the solution of the model in its different phases.
% In contrast to previous works, the compression we perform is \emph{exact}, meaning that it only resolves and simplifies redundancies in the TN at each step without loss of information.
% We illustrate the above with an application to the $3$-spin model, in which the average number of interactions per spin $\alpha$ controls transitions to different thermodynamic phases in the structure of the problem~\cite{mezard_alternative_2002}.
% We find that compressed TN contraction automatically implements the leaf removal algorithm~\cite{mezard_alternative_2002} and thus efficiently solves the problem when $\alpha < \alpha_d$, at which point a dynamical transition occurs.
% In contrast, compressed TN contraction scales superpolynomially when $\alpha > \alpha_d$ but improves substantially on the exponential scaling of TN contraction without compression.
% We further show that when $\alpha \in \left[2/3, 3/4 \right]$, compressed TN contraction outperforms naive GE.
% Finally, by devising a graphical scheme that exactly captures the dynamics of compressed TN contraction in the special case of spins appearing in exactly \emph{two} interaction terms, for which no leaf removal occurs, we show numerically that the TN method solves the problem in \emph{subexponential} time.